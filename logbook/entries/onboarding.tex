This chapter is a guide for new RAs to get started with the research infrastructure. 
It is a work in progress, and we welcome suggestions for improvement, 
especially from other RAs.
\textbf{It is not a substitute for the rest of the logbook, but rather a complement.}
Everything after this chapter overviews what you will need to learn.
Here we offer tips and tricks on how to learn it  – from the perspective of a new RA.

\section{Overarching Suggestions}

This section offers some high-level suggestions for 
how to structure your preparation and onboarding.
\begin{itemize}

\item The degree of preparation required depends on your own prior experience. 
If you have never used a command line before, written a \texttt{Makefile}, or used \texttt{git}/GitHub, 
then your learning curve is steeper.
So, we strongly suggest you do an honest accounting 
of where your experience and existing abilities stand. 
If you are confident in your experience, 
then you may be able to skim this document and jump right in.
If you are less confident, 
then you might want need to spend more time behind-the-scenes to be ready on your first day. 
Ultimately, both paths will lead to the same place with sufficient time and effort. 
The main author of this chapter had little to no prior experience
 with most of the technical tools described here.
He will be the first to tell you that your lack of experience 
is only an obstacle if you allow it to be.

\item Be prepared to get stumped and make mistakes. 
We want to be clear that you are not expected to be fluent in these tools on day one.
You are expected, however, to be willing to learn and to be able to learn from your mistakes.
And the only way to do so is by seeking help when needed: 
from current or former RAs, from Josh, or from project co-authors.
The absolute worst thing you can do is to struggle in silence.

\item If you don't have examples already, 
ask Josh or another RA for all or a portion of a project. 
There is no substitute for hands-on experience, 
and the closest you can get to that before you start is by learning from real code. 
In a similar vein, we strongly encourage you to take an existing project 
and adapt it to our workflow. 
Were you an RA? Did you write a thesis? 
Take that data and code and get it on GitHub, 
take your Word document and write it in \LaTeX, and so on.


\item AI can be a remarkably helpful learning tool.
We all know that generative AI has many pitfalls, 
especially if relying on its output verbatim when coding.
But it is much stronger in interpreting existing code, 
making it a natural fit for learning. 
Say you are looking at an example \texttt{Makefile} and you don't know what a particular line does.
If you feed it to ChatGPT (4.0 as of this writing), 
it might interpret about 80\% of the line correctly.
But that 80\% is likely far higher than your own baseline understanding, 
and you can often fill in the remaining 20\% through interpolation.

\end{itemize}

\section{Specific Topics}

\subsection{Setup}

There are a few programs that you will need to implement our projects. 
Things like \texttt{git}, \texttt{make}, 
and other command line tools are automatically installed if you have a Mac or use Linux.
But you should install Stata, Julia, \LaTeX, and R, 
among others idiosyncratic to the demands of your project. 
You should also install a text editor, such as VS Code or Sublime Text.
I strongly recommend VS Code. 
Once you open VS Code, navigate to the top right 
and click the layout button that is divided horizontally into two sections.
This will open the Terminal at the bottom of the screen. 
This is \textbf{the} command line. 
Within VS Code, install as many extensions as you think relevant. 
Effectively every language has its own helpful extension, and there are also extensions for GitHub.

\subsection{How we Use \texttt{make}}

Section \ref{entry:make} is rich in detail about \texttt{make}. 
This is a brief summary of its main applications 
to our workflow and how I suggest approaching it.

\begin{itemize}
\item \texttt{Makefile}s consist of targets and prerequisites organized as a recipe. 
Ultimately, the goal is to generate an output made from different ingredients.
Say you wanted to sketch out the recipe for bread.
The left of the colon is the bread, 
and the right of the colon is the oven, flour, water, yeast, and salt.
To the right of the \texttt{|} are order-only pre-requisites, 
which describes anything that must exist prior to the execution of the recipe 
but that the recipe does not require to be up-to-date.
Note that the oven is included as a prerequisite. 
After all, we need something to bake the bread in.
The same is true for \texttt{Makefile}s: 
we have code that facilitates the conversion of inputs to outputs.
If we have two .csv files we want to merge into a single .dta file, 
we write code to do so. 
All \texttt{make} does is to link our inputs, our code, and our outputs together.
So, as you will see below in the examples, most recipes are very similar: 
\begin{lstlisting}[language=make]
    path/to/output: code_file path/to/input1 path/to/input2 | order-only pre-requisite
        $(LANGUAGE) $(automatic variables)
    \end{lstlisting}
You simply replace \texttt{LANGUAGE} with the type of code (\texttt{R}, 
\texttt{STATA}, \texttt{PYTHON}) and the automatic variables with whatever you need to pass to your scripts.
In our bread example, we would write:
\begin{lstlisting}[language=make]
    bread: oven flour water yeast salt | shelf
        $(preheat) $<
        $(mix) $^
        $(bake) $@
    \end{lstlisting}
Note that you can have multiple rules for a single target; 
bread is a multi-step process.
You can also have a single rule for multiple targets. 
If you were reheating leftovers, you might be able to reheat them all together. 
Of course, we need something to store these ingredients and/or outputs in, 
which is the purpose of the \texttt{shelf}.
All that matters is that the shelf exists before we start baking.
For real \texttt{Makefile}s, 
the shelf is equivalent to a folder where the files are linked or stored.
Since our tasks employ the same set of order-only pre-requisites, 
we have a \texttt{Makefile} (\texttt{generic.make}) that automates all this for you.

This is a very, very brief overview. Please read the \texttt{make} section in full.
In particular, the sections on notes and pattern matching are most relevant.

\item Learning \texttt{make} is not easy; in fact, 
it is quite possibly the hardest aspect of our workflow. 
Jonathan makes a few strong suggestions about the best ways to learn. 
I concur that \href{https://www.youtube.com/watch?v=_Ms1Z4xfqv4}{MIT's Missing Semester} is a great introduction.
And you should absolutely read the documentation and articles provided below.
But I've found the most effective way to learn 
is by reading and running existing \texttt{Makefile}s.
Because our workflow has universally adopted \texttt{make}, 
the expectation is that all your code will flow through a \texttt{Makefile}.
The silver lining of this is that our projects have jointly created hundreds of \texttt{Makefile}s. 
Use them! 
I learned \texttt{make} before starting by reading and running the \texttt{Makefile}s. 
Most tasks have reports or \texttt{readme.md}'s that describe what they are doing. 
Treat that task as a puzzle, and the \texttt{Makefile} as your codebreaker. 
By forcing yourself to work through what each component of the \texttt{Makefile} does, 
you will internalize the nuances of it without realizing it. 
Moreover, the more \texttt{Makefile}s you read, 
the deeper your repository of knowledge will be. 
In practice, you rarely have to write a recipe from scratch.
You just need to be able to recall where you saw a similar recipe 
and adapt it to your needs.

\item Ask Josh to give you access to the repository for 
\textit{Market Size and Trade of Medical Services}.
It has too many \texttt{Makefile}s to count. 
If you want a couple easier \texttt{Makefile}s, 
look at the \texttt{Makefile}s for the tasks 
\texttt{medicare\_GAF\_by\_geo} and \texttt{clean\_ACS\_data}.
Scale up gradually, and challenge yourself with harder \texttt{Makefile}s 
like \texttt{correlations\_exporter\_fe\_hospital\_quality} 
and especially \texttt{downloaddata}.
Your goal should be to be able to identify the purpose of each recipe 
and be able to write a similar recipe with different inputs, outputs, 
automatic variables, etc.
If you can do that, you are in a good spot to hit the ground running.

\end{itemize}

\subsection{\texttt{git} and GitHub}

This section focuses on the subjective aspects of \texttt{git} and GitHub. 
By that, I mean stylistic choices that are not required to use Git, 
but rather are best practices we've adopted. 
As an introduction, 
the logbook mentions \href{https://www.atlassian.com/git/tutorials}{Atlassian}. 
It is an outstanding introduction, 
covering most of what you will need on a day-to-day basis. 
I also recommend \href{https://missing.csail.mit.edu/2020/version-control/}{MIT's Missing Semester} 
for an audiovisual introduction.

\subsubsection{Git}
\begin{itemize}
    \item The Git section suggests using a GUI to ease the transition. 
          I'm not sure I agree. 
          Using the command line for git forces you to know and understand the git commands you issue. 
          I strongly suggest sticking with the command line to manage git while you learn. 
          If anything, it will be a prohibitive choice in the first few days/weeks, 
          but will generate increasing returns once you move past the early obstacles.
    \item Commit and push often, especially in your first few weeks.
Our workflow is decentralized.
There is no built-in method for monitoring your progress or for anyone to check in on you; 
unlike Dropbox, your work does not automatically sync. 
Committing early and often can help get ahead of bugs, correct poor habits, and, 
most importantly, prove you are actually working!
    \item Commit messages should be descriptive but curt. 
It's perfectly fine to use shorthand and abbreviations in commit messages, 
but make them topical. 
    \item Commit outputs under specific circumstances only. 
    Those that go in the logbook or paper, for example. 
    Rarely, if ever, do we suggest committing data files.
If you've done everything correctly, 
the data outputs will automatically be generated when one runs \texttt{make}.
    \item Do not run from merge conflicts! VS Code makes merge conflicts easy to resolve. 
    A quick summary: 
    First, merge or pull the two branches to generate the conflict. 
    The command line will tell you which file(s) is/are conflicted. 
    Navigate to them via the file explorer on the left and click ``open merge editor'' when prompted. Then, just click the version you want to keep.
    And, you are about done.
    \item VS Code's built in source control paired with the GitHub extension is a powerful combination.
    It will highlight all changes you've made since your last commit
     and allow you to revert them at will. 
    Use it liberally.
\end{itemize}

\subsubsection{GitHub}
This section goes ``tab-by-tab'' to mimick the GitHub interface on web browsers. 
We now use GitHub to manage code, tasks, and assignments.
\begin{itemize}
    \item The first tab is ``code.'' 
    Here you will see all open branches with all their contents, 
    with the default branch most often \texttt{main} or \texttt{master}. 
    You can use this to visually inspect branches, past commits, and project meta-data, 
    like the \texttt{README.md}, recent activity, and contributors.
    \item The second tab is ``issues.'' 
    This is where we track assignments. 
    You will be assigned (or will assign yourself) issues on an ongoing basis. 
    Some suggestions for working with issues:
        \begin{itemize}
            \item Use the discussion space to post issue-specific questions, bugs, 
            or \textit{preliminary} outputs that you want feedback on. 
            Do not post every output you generate. 
            Use your best judgment, 
            but we generally suggest posting outputs that will start or contribute to an offline discussion. 
            Otherwise, commit the code only.
            \item If posting outputs, PNGs show up embedded in the comment,
             while PDFs can be viewed but do not appear automatically.
            \item Note the issue number. 
            We use that to identify issues during synchronous meetings 
            and when linking other issues or pull requests. 
            \item It's not a bad idea to post updates on your progress 
            if you haven't generated results, outputs, or closed the issue quite yet.
        \end{itemize}
        \item The third tab is ``pull requests''. 
        This is where we review code. 
        This is the most complicated tab.
        \begin{itemize}
            \item PRs get assigned numbers just like issues. 
            \item Make sure to change the default title to something brief but informative. 
            \item Give a detailed description of the PR. 
            A complete description identifies what code has changed, 
            what outputs have changed, and what issue(s) this closes. 
            \item Use $\#<number>$ to automatically link to an issue in the text. 
            \item For both this and PRs, you can link to specific comments in other issues by navigating to that comment, 
            clicking the three dots, and copying the link.
            \item If not done automatically, 
            link the issue using the ``development'' option so that this PR automatically closes the linked issue.
            \item Assign yourself as the assignee and assign at least one reviewer. 
            Generally, at least one co-author should be a reviewer, 
            or at the very least, an experienced RA.
            \item The reviewer will request changes. 
            When they do, make their requested changes, commit them, 
            push them to the same branch, and go back to the PR.
            The requested changes will now appear on code blocks that have the outdated label. 
            You should respond to the request by indicating which commit addressed this. 
            Copy the commit link from the PR's list of commits, 
            comment a sentence with the link, then resolve the conversation.
            \item Once completed, go to the reviewers tab and 
            click the two arrows arranged in a circle – like the recycling symbol – to re-request review. 
            \item Once this process happens and the reviewers are satisfied, they will approve the PR. 
            You can now go to the bottom and click the green ``squash and merge'' button to close the PR, 
            then you can delete the old branch.
            \item Sometimes, you will get a merge conflict preventing squashing. 
            Simple ones can be resolved on GitHub using their prompts, 
            while others must be done via the command line. 
            Follow GitHub's instructions on which you need to use.
        \end{itemize}
        \item The next tab is actions. 
        These consist of automatic steps that GitHub takes every time a PR is opened or a review is requested.
        \begin{itemize}
        \item Our specific actions check \texttt{Makefile}s 
        to ensure there are no \texttt{make} errors on the branch that will be merged. 
        An error will prevent merging. 
        \item These run automatically. 
        If you need to check something, 
        you can run them manually using the workflow dispatch, 
        but this is not recommended unless you are debugging actions themselves.
        \item If an action fails, you will see that it has failed on the PR. 
        You can click on the failed action to see what \texttt{Makefile} caused the issue. 
    \end{itemize}
    \item The other tabs are less relevant. 
    The only one worth mentioning is Wiki, which is exactly what it sounds like.
    Some projects use it for literature and other supporting writing, some don't. 
\end{itemize}

\subsection{The Shell}
Learning the shell has its advantages, 
and harnessing the power of the terminal can significantly improve your workflow and efficiency.
But doing so takes time, 
especially given our reliance on a few commands with one or two use cases. 
More often than not, we can succeed by recycling old code with minor changes. 
We do this with \texttt{grep} and \texttt{sed}, 
two commands that you will see frequently. 
We recommend \href{https://missing.csail.mit.edu/}{MIT's Missing Semester} 
for an efficient introduction to the shell.
Of the other links the logbook mentions, 
\href{https://ryanstutorials.net/linuxtutorial/}{this} is my favorite. 
% If you want to expand your knowledge of the terminal beyond the basics....add any notes from Luke here.

\subsection{Managing Time}

An underrated aspect of working in a full-time research environment is decentralization.
You will have multiple tasks, projects, 
and organizing your time and priorities is often left to you.
Here are some very simple guidelines as you start to take on more responsibility.
\begin{itemize}
    \item Prioritize substantive, economic tasks over bug fixing and workflow optimization. 
         There are always exceptions – for example, a bug prevents your paper from compiling. 
         But in general, you should give more weight to economically significant tasks. 
         Workflow optimizing can occur before, during, or after submission. 
         Economic tasks cannot. 
         This is not to say you should neglect workflow optimization. 
         It is to say that you should be mindful of the marginal benefit of each task.
    \item Prioritize tasks that are time-sensitive.
          Not all tasks have a formal or even informal deadline.
          But if you are working on two projects and one has a team meeting tomorrow, 
          then you should prioritize that project.
    \item When in doubt, ask what matters. 
          Frequent communication is a hallmark of our workflow and a prerequisite for success.
          If you are unsure of what matters more, ask during a meeting, on GitHub, or via email.
\end{itemize} 

\subsection{Previous RPs' Contact Information}

Although the old RPs will have departed or will soon depart by your start date, 
they are still a valuable resource for questions, 
whether it be specific code they wrote, general workflow questions, or acclimating to Chicago.
Below is a list of their contact information and the projects they worked on. 
% ideally, update this before every new RP class.
\begin{itemize}
    \item Scott Blatte: \texttt{sblatte@gmail.com}, \texttt{sblatte@uchicago.edu}, 617-548-7905. \textbf{Projects:} \texttt{doctor-tasks}, \texttt{health-care-jobs}.
    \item Nick Niers:
    \item Xiaoyang Zhang: \texttt{xiaoyangzhang@uchicago.edu}.  \textbf{Projects:} \texttt{travel-nursing}.
    \item Luke Motley: \texttt{ljmotley@gmail.com}, 618-972-6062. \textbf{Projects:} \texttt{doctor-tasks}, \texttt{health-care-jobs}, \texttt{air-quality}, \texttt{hospital-tax}, \texttt{GHHO}
    \item Dylan Baker: \texttt{dylanrbaker239@gmail.com}, 859-396-0663. \textbf{Projects:} \texttt{travel-nursing}, \texttt{health-care-jobs}, \texttt{voting-aca}
    \item Vaidehi Parameswaran: \texttt{vp2409@columbia.edu}. \textbf{Projects:} \texttt{doctor-tasks}, \texttt{travel-nursing}
\end{itemize}




